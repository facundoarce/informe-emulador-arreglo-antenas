\documentclass[class=article, crop=false]{standalone}

\begin{document}

\todo[inline]{Escribir un resumen}
%
% En este trabajo se propuso el desarrollo de un sistema que permita emular las señales de radiofrecuencia recibidas en una estación terrena provenientes de satélites en órbita baja. De éste modo, se simularán las señales recibidas por parte de un arreglo plano de antenas. Deberán considerarse, al menos, los efectos de atenuación por la distancia y el efecto Doppler a lo largo del tiempo de vista entre el satélite y la estación terrena en función de los parámetros orbitales.
%
% Para la generación de las señales emuladas se propone la utilización de un dispositivos de síntesis digital directa (DDS) AD9959 de la empresa Analog Devices para implementar apuntamiento electrónico de haz. Esta tecnología permite transmitir una señal de radiofrecuencia en diferentes direcciones sin la necesidad de apuntar físicamente la antena. Se realizarán ensayos con un prototipo de 4 canales y se deberá estudiar la arquitectura para extender la capacidad a 4xN canales.
%
%
% ----------------------------------------
%
% Se desarrolló un emulador de las señales de RF provenientes de satélites en órbita baja recibidas por parte de un arreglo de antenas. Se modeló la señal recibida por los elementos del arreglo en función de la geometría del arreglo y la trayectoria del satélite. Se desarrolló un prototipo que genera dichas señales para un arreglo lineal de 4 elementos. Se estudió la posibilidad de extender el prototipo a un arreglo planar de 4xN elementos.

\end{document}
