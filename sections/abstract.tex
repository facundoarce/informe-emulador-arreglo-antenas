\documentclass[class=article, crop=false]{standalone}

\begin{document}

Antenna arrays for communication with satellites in low Earth orbit are widely spread.
The possibility of using beamforming techniques to track satellite trajectories makes them interesting objects of study.
The need of a physical antenna array can be supressed by means of an emulator of the signals received by the array elements.
These signals can be then used as inputs to other devices such as a data acquisition and processing system.

The goal of the present work is to develop a system that emulates the signals received by an antenna array.
A model of these signals for a linear array was conceived and a protoype that generates them was developped with the possibility to extend its capabilities to a planar array.

% Los arreglos de antenas para comunicaciones con sátelites de órbita baja son ampliamente utilizados.
% Su capacidad de apuntamiento de haz para el seguimiento de órbitas sátelitales los convierte en interesantes objetos de estudio.
% La necesidad de contar con un arreglo de antenas físico y de tener que esperar la pasada de un satélite puede suprimirse teniendo un emulador de las señales recibidas por los elementos del arreglo.
% Estas señales pueden utilizarse como entradas para otros dispositivos como por ejemplo un sistema de adquisición y procesamiento de datos.
%
% El objetivo del presente trabajo es desarrollar un sistema que permita emular las señales recibidas por un arreglo de antenas.
% Para ello, se modelaron las señales recibidas por los elementos de un arreglo lineal y se desarrolló un prototipo que genera dichas señales con posibilidad de extenderlo a futuro a un arreglo plano de antenas.

\end{document}
