\documentclass{article}
\newenvironment{standalone}{\begin{preview}}{\end{preview}}
\usepackage{../includes}

\begin{document}
\begin{standalone}
  \section{Conclusiones}

  Se modelaron las señales recibidas por los elementos de un arreglo lineal de antenas para señales emitidas por un satélite en órbita baja.
  Para esto se estudió el movimiento del satélite, la señal emitida y los fenómenos que modifican la señal recibida por los elementos del arreglo teniendo en cuenta el corrimiento de frecuencia por efecto Doppler, los desfasajes producidos por la geometría del arreglo y la diferencia de amplitud por las pérdidas en espacio libre.

  Se implementó un prototipo de un emulador de arreglo de antena de 1x4.
  Este prototipo consiste de un microcontrolador y un sintetizador digital directo.
  El microcontrolador calcula las señales a emular de acuerdo con el modelo estudiado y se comunica con el sintetizador para darle los comandos de frecuencia, fase y amplitud.
  El sintetizador recibe esos comandos y entrega a la salida las señales deseadas.

  Se comprobó que dichas señales se correspondan con el modelo propuesto mediante el análisis de las señales generadas en un osciloscopio y un analizador de espectro.
  Con el analizador de espectro se realizó además una breve caracterización de la respuesta en frecuencia del sintetizador.
  Por último, también se comprobó el modelo mediante la realización de un apuntamiento de haz mediante cables de distintas longitudes, induciendo así retardos en las señales.

  Se deja como trabajos a futuro, la mejora del modelo propuesto, como la inclusión de la modulación de la señal emitida, la consideración de una órbita real del satélite y de otras pérdidas, no sólo por espacio libre.
  Se propone además, extender el prototipo realizado a un arreglo de Nx4, con la utilización de N sintetizadores.

\end{standalone}
\end{document}
