% \documentclass[class=article, crop=false]{standalone}

\documentclass{article}
\newenvironment{standalone}{\begin{preview}}{\end{preview}}
\usepackage{../includes}

\begin{document}

Para establecer la comunicación de un satélite de órbita baja con una estación terrena, se requiere contar con una antena con capacidad de apuntamiento para realizar el seguimiento del satélite durante su paso sobre la estación terrena. Una tecnología que puede emplearse para evitar el uso de sistemas mecánicos de seguimiento es el apuntamiento electrónico de haz utilizando un arreglo de antenas y un conformador de haz.

Para ensayar el conformador de haz durante su desarrollo, en principio, debería disponerse de un arreglo de antenas y de la capacidad de recibir señales desde diferentes orientaciones.
Para evitar esto, se puede desarrollar un emulador de arreglo de antenas que genera las señales que reciben los diferentes elementos del arreglo durante la fase de recepción.
Estas señales pueden usarse como entradas para sistemas de conformación de haz de diferentes tecnologías, ya sean ópticos, analógicos o digitales.

En el presente trabajo se desarrolló un emulador para un caso simple, esto es, un arreglo lineal de antenas y un satélite que pasa por el cenit de la estación terrena describiendo una órbita circular.
Se modelaron las señales recibidas por los elementos del arreglo teniendo en cuenta las variaciones de fase por la geometría del arreglo, de frecuencia por efecto Doppler y de amplitud por pérdidas por trayectoria en espacio libre.
Se implementó el emulador con un microcontrolador y un sintetizador digital directo (DDS) y se realizaron ensayos con un osciloscopio y un analizador de espectro para validar el modelo propuesto y la implementación realizada.
Finalmente, se estudió la arquitectura del DDS elegido para extender la capacidad del emulador para arreglos planos de antenas.

\end{document}
