\documentclass[class=article, crop=false]{standalone}

\begin{document}

Arreglos de antenas y técnicas de apuntamiento de haz pueden utilizarse para comunicaciones satelitales.
Es posible realizar estudios de estos sistemas y suprimir la necesidad de contar con un arreglo de antenas físico y de tener que esperar la pasada de un satélite si se cuenta con un emulador de las señales recibidas por los elementos del arreglo.
Se pueden utilizar estas señales como entradas para otros dispositivos como por ejemplo un sistema de adquisición y procesamiento de datos.

El objetivo del presente trabajo es desarrollar un sistema que permita emular las señales recibidas por un arreglo de antenas.
Para ello, se modelaron dichas señales para un arreglo lineal, se desarrolló un prototipo que las genera y se estudió la posibilidad de extender el emulador para un arreglo plano de antenas.

\end{document}
